\chapter{How it does it}

This chapter explains in detail how the program {\tt pphs} was implemented
from a programmer's viewpoint.  It was implemented in the C programming
language, as this is a commonly used language often used for writing UNIX tools.
The program code is shown in Appendix~\ref{prog-code} and the makefile in
Appendix~\ref{make-code}.

\section{General sequence of events}

When the {\tt pphs} program is run, the program first finds out what, if any,
options it has been called with.  If any have been specified, the appropriate
variables are set.  The program then checks it has been called with exactly one
further argument.  If not, the program terminates with an
explanatory error message.  If called correctly, the program then checks that the
supplied argument is the name of a file that exists and is readable.
The program is normally used
on files ending with a {\tt .hs} extension.  When called with a filename
with no extension and that file is not found, then it appends the extension and searches
for that file.  If no file with that name is found or the file is unreadable, an
error message is produced and the program terminates.  If the file is found, the
program starts the typesetting process by writing out the opening
\LaTeX\ command to {\tt stdout}.
This defines the \LaTeX\ environment which the program exploits to do the typesetting.
It then initialises the variables used in the program.

This done, the first character is read.  The program enters a loop and keeps
reading characters until the end of the file is reached.  As each character is read
in, its typeface is established and it is stored with its typeface in something
called the {\em line store\/}.  If any left indentation is
encountered, the correct characters to be skipped are identified from the {\em left
indentation stack} and copied into the line store.  Internal alignment is checked
for and if any is found, appropriate variables are set accordingly.  Each stored line is
added to both the left indentation stack and the {\em writing queue}.  When the value of the
internal alignment changes, or it has been established that the first line in the writing
queue is not part of any internal alignment section, the lines in the queue are written out.

Once all the lines are written out, {\tt pphs} then writes the closing \LaTeX\ command
and terminates.

\section{Basic storage unit for a line of code} \label{line-store}

The basic storage unit used in {\tt pphs} is the line store unit.
This stores the details of one line of Haskell code.  These are
the characters on the line, the typeface associated with each
character, the length of the line, the indentation level and the position of
any internal alignment in the line.

In the C program, {\tt ElementType} is the structure used for this type.  This has
five parts:
\begin{itemize}
\item {\tt chars} which stores the characters used on the line of Haskell
code

\item {\tt typeface} which stores the typeface values associated with the
characters on that line

\item {\tt indentation} which stores the level of the line's indentation

\item {\tt length} which stores the length of the line

\item {\tt col} which stores the column where any internal alignment occurs or
is set to {\tt 0} if there is none
\end{itemize}
The variable {\tt store} in the main program is of type {\tt ElementType} and
is used as the basic storage unit for the current line.  Its C declaration is
\begin{quote}
\begin{verbatim}
typedef struct ElementType_Tag {
  char chars[MAXLINELENGTH];
  enum face typeface[MAXLINELENGTH];
  int indentation, length, col;
} ElementType;
\end{verbatim}
\end{quote}

\section{Stack of lines for left indentation}

Due to \LaTeX 's variable width characters, {\tt pphs} cannot simply uses spaces
for the left indentation as in the input Haskell file.  It has to work out how far
each line is indented by finding out what it is indented under.  As each line is
completed, it is added to a stack of lines, each line being stored in a basic
storage unit.  If the line at the top of the stack is of a greater or equal
indentation level and of a lesser or equal length, then it is no
longer required for calculating typeset indentation
and can be disposed of.  Once all lines of greater indentation level have been removed
from the top of the stack, the current line can then be added.

When a line's indentation level, in terms of the number of spaces used in the
input, has been determined, {\tt pphs} has to find
out the characters that determine the actual typeset length of the indentation.  To get this,
{\tt pphs} looks down the stack until it comes to a line whose indentation is less than
that of the current line and whose length is greater than the indentation level of the
current line.  Once a suitable line is found, its characters and typefaces are copied
into the line store of the current line; then the rest of the current line is read in,
overwriting the characters beyond the indentation level.  If there is no line preceding
the current one that is as long as the indentation level of the current line, spaces
are placed in the line store instead.

A special case has been made for left indentation.  Most of the time, the left-hand edge
of the characters will be aligned, but where a {\tt |} is aligned under an {\tt =} sign, it is
centered under the sign.  This will be the case for any further {\tt |} symbols aligned
under this {\tt =} sign.

The type {\tt StackType} is used in the program for the stack.  This makes a stack of
the basic line storage units of {\tt ElementType}, together with a set of functions available
for use with stacks.  These are {\tt CreateStack}, which returns an empty stack;
{\tt IsEmptyStack}, which returns {\tt 1} if the stack which it is called with is empty,
{\tt 0} otherwise; {\tt Push}, which takes a stack and an element and returns the stack
with the element pushed onto the top; {\tt Top}, which takes a stack and returns the top
element of the stack; {\tt Pop}, which takes a stack and returns it with the top element
removed; and {\tt PopSym}, which is the same as {\tt Pop} except that it does not free the
memory used by the top element - this function was found necessary to fix a fault caused by
returning to a stack's previous state, having popped off elements in the interim period.

\section{Internal alignment identification}

Internal alignment is deemed to have occurred when a character matches the one
immediately above it, the preceding characters in both lines are spaces, and there is
more than one space preceding the character on at least one of the lines.

To check for this in {\tt pphs}, the current position on the line, indicated by
the linecounter, must be greater than one because either the current line or
the previous line will be required to have two spaces before the current position.  The current
line will be located in the line store and the previous line will be at the rear of the queue
of lines waiting to be written out.

One special case has been implemented for internal alignment.  This is to allow Haskell
type declarations, such as in the example below, to align with their corresponding function
definitions.
\begin{quote}
\begin{verbatim}
fst            :: (a,b) -> a
fst (x,_)       = x
\end{verbatim}

\end{quote}
The {\tt =} sign can be under either the first or second {\tt :} symbol for the
internal alignment to be recognised.

\section{Typefaces and mathematical characters}

Each character has a typeface value associated with it.  Normally, this will
indicate the type of token the character is part of, either keyword, identifier,
string, comment, number or maths symbol, but where Haskell uses an ASCII character
simulation of a mathematical character or some other special symbol, the typeface
value will indicate this as well.

In the program, the typeface values are of the
enumerated type called {\tt face}, which has the values shown in Table~\ref{tf-val}.
They are used in the basic storage unit {\tt ElementType} in the {\tt typeface} part.

\begin{table}
\begin{center}
\begin{tabular}{|c|l|} \hline
{\em value\/} & {\em indicates\/} \\ \hline
{\tt KW} & keyword \\
{\tt ID} & identifier \\
{\tt IE} & exponent identifier \\
{\tt ST} & string \\
{\tt SE} & exponent string \\
{\tt CO} & comment \\
{\tt CE} & exponent comment \\
{\tt NU} & number \\
{\tt NE} & exponent number \\
{\tt MA} & maths \\
{\tt SP} & space \\
{\tt LC} & line comment \\
{\tt RC} & regional comment begin \\
{\tt CR} & regional comment end \\
{\tt BF} & backwards/forwards quote \\
{\tt FQ} & forwards quote \\
{\tt EQ} & escape quote \\
{\tt DQ} & double quote begin \\
{\tt QD} & double quote end \\
{\tt EE} & escape double quote \\
{\tt DC} & second part of double character \\
{\tt DP} & double plus \\
{\tt CP} & colon plus \\
{\tt LE} & less than or equal to \\
{\tt GE} & greater than or equal to \\
{\tt LA} & left arrow \\
{\tt RA} & right arrow \\
{\tt RR} & double right arrow \\
{\tt TI} & times \\
{\tt EX} & double exponent character \\
{\tt XP} & exponent \\
{\tt BE} & bar aligned under equals \\ \hline
\end{tabular}
\end{center}
\caption{Typeface values} \label{tf-val}
\end{table}

\subsection{Current character and retrospective update}

The {\tt pphs} program has to determine the typeface of a character without knowledge of the
characters to follow.  Therefore it allocates the value depending on the status
of various boolean variables.  This may subsequently be found to be wrong once the remaining
characters of that token have been read.

In the case of keywords and double characters, these are only identifiable
as such once all the characters of the token have been read in.  Having established
the existence of a keyword or double character, {\tt pphs} then goes back and changes
the typeface values for the appropriate characters.

The functions {\tt CheckForDoubleChar} and {\tt CheckForKeyword} perform this in the
program.

\section{Writing lines out}

Lines are written to {\tt stdout}, but not immediately on being read in.  Instead they
are held back while it is established whether or not they form part of a section of
internal alignment.

Before any typeset Haskell code is written, {\tt pphs} writes an opening \LaTeX\ command
{\tt \char'134 begin\char'173 tabbing\char'175 } to {\tt stdout}.  This defines the
\LaTeX\ environment that the typeset code will be written in.  At the end,
{\tt \char'134 end\char'173 tabbing\char'175 } is written to terminate this
environment.

\subsection{The line queue}

Lines are stored in a queue while they are waiting to be written out.  
The elements of the queue are the basic line storage units described in
Section~\ref{line-store}.

In the program, the queue is of type {\tt QueueType}
and a set of functions related to queues is available.  This set consists of
{\tt CreateQueue}, which returns an empty queue; {\tt IsEmptyQueue}, which takes
a queue and returns {\tt 1} if the queue is empty, {\tt 0} otherwise; {\tt LengthOfQueue},
which takes a queue and returns its length; {\tt FrontOfQueue}, which takes a queue and
returns a pointer to its front element; {\tt RearOfQueue}, which takes a queue and returns
a pointer to its rear element; {\tt AddToQueue}, which takes a queue and an element and
returns the queue with the element added to the rear; {\tt TakeFromQueue}, which takes
a queue and returns the queue with the front element removed.

The last line in the queue is inspected to search
for internal alignment; if any is found, the internal alignment variable of that
line is altered accordingly.

\subsection{When lines are written}

The queue is written out by the function {\tt WriteQueue} when a section of internal
alignment is commenced or terminated
or when it has been established that there is no internal alignment involving the first line
in the queue.  If the section being written out has been found to have
no internal alignment, then the last line is retained
in the queue because it may form part of the next section of internal alignment.

At the end of the input, {\tt WriteRestOfQueue} writes all the lines remaining in the queue.
This is because the last line of Haskell code will not form part of any further section of
internal alignment and can therefore be written out.  Facilities
are provided in the function {\tt WriteLine} to avoid writing the last newline
character at the end of the Haskell
file, as this would create an unwanted blank line in the final document.

\subsection{Writing a line}

The function {\tt WriteLine} is used in {\tt pphs} to write out one line.  This is
called from either {\tt WriteQueue} or {\tt WriteRestOfQueue} and is supplied with
a basic line storage unit containing the line needing to be written out together with a
flag stating whether or not a \LaTeX\ newline character is required.

If a line has any left indentation, this is written out first by calling the function
{\tt WriteSkipover}.  The rest of the line is then written out by {\tt WriteWords}
followed if necessary by the newline character.  Both these functions are given
the current line in the line store.

\subsection{Writing left indentation}

As \LaTeX\ uses variable width characters, fixed width spaces cannot be used for the
left indentation.  Instead, the width of the characters above the current line needs
to be skipped.  The {\tt \char'134 skipover} command, defined in the {\tt pphs.sty}
style file (see Section~\ref{style-file}), is used by the function {\tt WriteSkipover}
to get \LaTeX\ to do this.  The command is supplied with the typefaces and characters
in the lines above, and, with this, \LaTeX\ creates the correct amount of
indentation in the typeset result.  The typefaces and characters are written in
braces as the argument to {\tt \char'134 skipover} by calling {\tt WriteStartFace},
{\tt WriteChar}, {\tt WriteSpaces} and {\tt WriteFinishFace}.  The typeface functions
are called with the typeface value whereas the other two are given the line store,
current position and where the end of the skipover section is.

Using this specially defined {\tt \char'134 skipover} command avoids having to get
information back from \LaTeX , therefore keeping the information flow unidirectional.

\subsection{Writing the rest of a line}

The function {\tt WriteWords} writes out the indented line once any left indentation
has been dealt with.  Starting at the indentation level of the line, it uses the functions
{\tt WriteStartFace}, {\tt WriteChar}, {\tt WriteSpaces} and {\tt WriteFinishFace} to
write out each character and its typeface.  The typeface functions are called with
the typeface value whereas the other two are given the line store, current position
and where the end of the line is.

\subsection{Writing \LaTeX\ typeface commands}

Every character has a typeface associated with it, so at the start and finish of every
line and every time the current typeface changes, typeface commands have to be written
out.  This is done by the functions {\tt WriteStartFace} and {\tt WriteFinishFace}.
They write the appropriate \LaTeX\ typeface commands according to the typeface values
given as shown in Table~\ref{tf-comms}.  To avoid complications, double characters have
their typefaces written out as part of the character command, therefore they need no
further typeface commands.  Similarly, the user-redefinable quote mark characters
have their typeface defined in their definitions, so do not need any more typeface
commands.

\begin{table}
\begin{center}
\begin{tabular}{|c|l|l|} \hline	% ``commands'' to be over two columns
{\em value\/} & \multicolumn{2}{c|}{\em commands\/} \\ \cline{2-3}
              & {\em begin\/} & {\em end\/} \\ \hline
{\tt KW} & {\tt \char'173 \char'134 keyword} & {\tt \char'134 /\char'175 }\\
{\tt ID} & {\tt \char'173 \char'134 iden} & {\tt \char'134 /\char'175 }\\
{\tt IE} & {\tt \char'173 \char'134 iden} & {\tt \char'134 /\char'175 \$ }\\
{\tt ST} & {\tt \char'173 \char'134 stri} & {\tt \char'134 /\char'175 }\\
{\tt SE} & {\tt \char'173 \char'134 stri} & {\tt \char'134 /\char'175 \$ }\\
{\tt CO} & {\tt \char'173 \char'134 com} & {\tt \char'134 /\char'175 }\\
{\tt CE} & {\tt \char'173 \char'134 com} & {\tt \char'134 /\char'175 \$ }\\
{\tt NU} & {\tt \char'173 \char'134 numb} & {\tt \char'134 /\char'175 }\\
{\tt NE} & {\tt \char'173 \char'134 numb} & {\tt \char'134 /\char'175 \$ }\\
{\tt MA} & {\tt \$ } & {\tt \$ }\\
{\tt SP} & & \\
{\tt LC} & & \\
{\tt RC} & & \\
{\tt CR} & & \\
{\tt BF} & &  \\
{\tt FQ} & &  \\ \hline
\end{tabular} \hskip3mm \begin{tabular}{|c|l|l|} \hline
{\em value\/} & \multicolumn{2}{c|}{\em commands\/} \\ \cline{2-3}
              & {\em begin\/} & {\em end\/} \\ \hline
{\tt EQ} & & \\
{\tt DQ} & & \\
{\tt QD} & & \\
{\tt EE} & & \\
{\tt DC} & & \\
{\tt DP} & & \\
{\tt CP} & & \\
{\tt LE} & & \\
{\tt GE} & & \\
{\tt LA} & & \\
{\tt RA} & & \\
{\tt RR} & & \\
{\tt TI} & {\tt \$ } & {\tt \$ } \\
{\tt EX} & {\tt \$ } & \\
{\tt XP} & {\tt \$ } & \\
{\tt BE} & & \\ \hline
\end{tabular}
\end{center}
\caption{Typeface values and related \LaTeX\ commands} \label{tf-comms}
\end{table}

\subsection{Writing characters}

{\tt WriteChar} is the function which handles writing characters.  It takes the line store,
the current position on the line and the end of the current section - either the skipover
section or the writing section - and returns the current position on the line which will
have been incremented if a double character has been written.  If the first character of
a double character is the last character of a skipover section, it will not be written
so the indentation for that line will fall instead, below the start of the double
character in a line above.  Most characters are written out as they were inputted,
but many require special \LaTeX\ code.

As \LaTeX\ uses embedded typesetting commands, some characters are reserved for this
purpose.  Should any of these characters appear in the input Haskell code, {\tt pphs}
has to produce the appropriate \LaTeX\ code to avoid these characters upsetting the typesetting
process.  The characters and the replacement \LaTeX\ code are shown in Table~\ref{rep-chars}.
\begin{table}
\begin{center}
\begin{tabular}{|c|l|} \hline
{\em input\/} & {\em \LaTeX\ code output } \\ \hline
{\tt \#} & {\tt \char'134 \#} \\
{\tt \$} & {\tt \char'134 \$} \\
{\tt \%} & {\tt \char'134 \%} \\
{\tt \&} & {\tt \char'134 \&} \\
{\tt \char'176 } & {\tt \char'134 char'176 } \\
{\tt \_} & {\tt \char'134 \_} \\
{\tt \char'134} & {\tt \char'134 hbox\char'173 \$setminus\$\char'175 } \\
{\tt \char'173} & {\tt \char'134 hbox\char'173 \$\char'134 cal \char'134 char'146 \$\char'175 } \\
{\tt \char'175} & {\tt \char'134 hbox\char'173 \$\char'134 cal \char'134 char'147 \$\char'175 } \\
{\tt *} & {\tt \char'134 times}\\ \hline
\end{tabular} \hskip3mm \begin{tabular}{|c|l|} \hline
{\em input\/} & {\em \LaTeX\ code output } \\ \hline
{\tt ++} & {\tt \char'134 plusplus}\\
{\tt :+} & {\tt \char'173 :\char'175 \char'173 +\char'175}\\
{\tt <=} & {\tt \$\char'134 leq\$}\\
{\tt >=} & {\tt \$\char'134 geq\$}\\
{\tt <-} & {\tt \$\char'134 leftarrow\$}\\
{\tt ->} & {\tt \$\char'134 rightarrow\$}\\
{\tt =>} & {\tt \$\char'134 Rightarrow\$}\\
{\tt \char'173 -} & {\tt \char'173 \char'134 com \char'134 \char'173 -\char'134 /\char'175 }\\
{\tt -\char'175 } & {\tt \char'173 \char'134 com -\char'134 \char'175 \char'134 /\char'175 }\\
{\tt --} & {\tt \char'173 \char'134 rm -\char'175 \char'173 \char'134 rm -\char'175 }\\ \hline
\end{tabular}
\end{center}
\caption{Haskell input and replacement \LaTeX\ code} \label{rep-chars}
\end{table}

When a mathematical character needs written, {\tt WriteChar} outputs the \LaTeX\ code for
the character rather than the Haskell ASCII character simulation.  Some of these
simulations use more than one character, so this could cause problems if some left
indentation is aligned under the second character of such a simulation.  It has been
decided that in this case, the output from {\tt pphs} will cause the indented line
to align under the start of the double character rather than the centre or end of it.
The Haskell ASCII simulations and the \LaTeX\ codes that replaces them are shown in
Table~\ref{rep-chars}.  The non-standard command {\tt \char'134 plusplus} is defined
in the {\tt pphs.sty} style file (see Section~\ref{style-file}).

When a {\tt |} symbol is aligned under an {\tt =} sign at the left indentation,
{\tt \char'134 bareq} is output.  This command is defined in the {\tt pphs.sty}
style file explained in Section~\ref{style-file} and causes \LaTeX\ to write the bar symbol
centrally in the space it would have taken to write an equals sign, thereby causing
the bar to be positioned centrally under the equals sign it is aligned under and the text
following the bar to align with that after the equals sign.

For writing spaces, {\tt WriteSpaces}, called with the line store, current position and the
position of the end of the current section, first counts the number of consecutive spaces
to be written before writing out a {\tt \char'134 xspa} command with an argument of
the number of spaces needed.  This makes the output code easier to read.  The
{\tt \char'134 xspa} command is defined in the {\tt pphs.sty} style file explained
in Section~\ref{style-file}.  Any tab characters are treated as spaces by {\tt pphs}
with the number of spaces they represent being calculated from the current position
on the line and the {\tt tablength} variable, which may have been changed from its
default of 8 by the {\tt -t} option at the program call.

Numbers are written by {\tt WriteChar}, including floating point numbers.

As \LaTeX\ provides several different quote marks, it was decided that the user
should be able to choose a preferred symbol.  An input quote mark {\tt '} can
either be a prime or a quote mark in the output.  This requires the program to
determine which it is.  In program code this is fine, but in comments or strings
the marks won't necessarily be used in a manner from which it can easily be
determined which symbol is required.  In program code, an input {\tt '} is deemed
to be a quote mark if either it is preceded by punctuation or a quote has
already been opened; otherwise it is a prime.  Of the quote marks, these can
either be for actual quotes or an escape quote where a quote mark is being quoted.
Special cases has been implemented when the input file contains a quote within a comment
started with a backquote and ended with a forwards quote, and for \LaTeX\ style
quotes in comments started with two backquotes and ended with two forwards quote
marks.  All input {\tt '} in strings, other than escape quotes, are treated
as primes.  In strings, an input {\tt '} may be an apostrophe, however, there is
little way of telling this.\label{string-apostrophe}  One of five different pieces
of \LaTeX\ code can be produced having received {\tt '} as input.
\begin{itemize}
\item {\tt \char'134 forquo} for a forwards quote mark
\item {\tt \char'134 escquo} for an escape (quoted) quote mark
\item {\tt \char'173 \char'134 com '\char'134 /\char'175 } for a forward quote ending a quote
in a comment opened by a backquote
\item {\tt \char'173 \char'134 com ''\char'134 /\char'175 } for two forward quotes ending a quote
in a comment opened by two backquotes
\item {\tt '} for a prime which will be in the math font
\end{itemize}
The first two are commands defined in the {\tt pphs.sty} style file and are
thus user-redefinable as described in Section~\ref{user-adj}.  Backquotes, input
as {\tt `}, are either in the comment typeface for backquotes in comments or in
math font elsewhere.

\subsection{Writing internal alignment}

To commence a section of internal alignment, either of the functions {\tt WriteQueue}
or {\tt WriteRestOfQueue} write out
{\tt \char'134 begin\char'173 tabular\char'175 \char'173 @\char'173 \char'175 l@\char'173 \char'134 xspa1\char'175 c@\char'173 \char'175 l\char'175 }
before writing the first line of the section.  This provides an environment
with three columns.  The first column accommodates the Haskell code to the left of the
internal alignment, the second has the symbols that line up vertically, while the third
has the Haskell code to the right.  The Haskell code is written complete with its \LaTeX\
typesetting commands with the addition of {\tt \&} symbols denoting the breaks between
columns.  Once the internal alignment section has been completed, the
{\tt \char'134 end\char'173 tabular\char'175 } command is written to terminate the
environment.