\documentstyle[11pt,slpj,abstracts]{article}

\begin{document}

\title{Abstracts of GRIP/GRASP-related papers and reports, 1990
}

\author{The GRASP team \\ Department of Computing Science \\
University of Glasgow G12 8QQ
}

\maketitle

\begin{abstract}
We present a list of papers and reports related to the GRIP 
and GRASP projects,
covering {\em the design, compilation technology,
and parallel implementations of functional programming languages, especially
\Haskell{}}.

Most of them can be obtained by FTP.  Connect to {\tt ftp.dcs.glasgow.ac.uk},
and look in {\tt pub/glasgow-fp/papers}, {\tt pub/glasgow-fp/drafts}, {\tt pub/glasgow-fp/tech\_reports},
or {\tt pub/glasgow-fp/grasp-and-aqua-docs}.

They can also be obtained by writing to 
Alexa Stewart, Department of Computing Science,
University of Glasgow G12 8QQ, UK.   Her electronic mail address is
alexa@dcs.glasgow.ac.uk.
\end{abstract}

\section{Published papers}

\reference
{Philip Wadler}
{Comprehending monads}
{{\em ACM Conference on Lisp and Functional Programming},
Nice, France, pp.\ 61--78, June 1990.}
{
Category theorists invented {\em monads\/} in the 1960's
to concisely express certain aspects of universal algebra.
Functional programmers invented {\em list comprehensions\/}
in the 1970's to concisely express certain programs involving lists.
This paper shows how list comprehensions may be generalised
to an arbitrary monad, and how the resulting programming feature
can concisely express in a pure functional language some
programs that manipulate state,
handle exceptions, parse text, or invoke continuations.
A new solution to the old problem
of destructive array update is also presented.
No knowledge of category theory is assumed.
}

\reference
{Philip Wadler}
{Linear types can change the world!}
{{\em IFIP TC 2 Working Conference on Programming
Concepts and Methods}, Sea of Galilee, Israel, April 1990.}
{
The linear logic of J.-Y.~Girard suggests a new type
system for functional languages, one which supports operations
that ``change the world''.
Values belonging to a linear type must be used exactly once:
like the world, they cannot be duplicated or destroyed.
Such values require no reference counting or garbage collection,
and safely admit destructive array update.
Linear types extend Schmidt's notion of single threading;
provide an alternative to Hudak and Bloss' update analysis;
and offer a practical complement to Lafont and Holmstr\"om's elegant
linear languages.
}

\reference{K Hammond and SL Peyton Jones}
{Some early experiments on the GRIP parallel reducer}
{Proc Nijmegen Workshop on Parallel Implementations of Functional Languages, TR 90-16, Dept
of Informatics, University of Nijmegen, ed Plasmeijer, 1990, pp51-72}
{
GRIP is a multiprocessor designed to execute functional programs in 
parallel using graph reduction.  We have implemented a compiler for
GRIP, based on the Spineless Tagless G-machine
and can now run parallel functional programs with substantial absolute
speedup over the same program running on a uniprocessor Sun.

Parallel functional programming shifts some of the burden of resource
allocation from the programmer to the system.  Examples of such
decisions include: when to create a new concurrent activity (or {\em
thread}), when to execute such threads, where to execute them, and so
on.

It is clearly desirable that the system should take such decisions,
{\em provided it does
a good enough job}.  For example, a paged virtual memory system
almost always does an adequate job, and a programmer very seldom
has to intefere with it.
The big question for parallel functional programming is whether good
resource-allocation strategies exist, and how well they perform under a 
variety of conditions.

Now that we have an operational system, we are starting to carry out
experiments to develop resource-allocation strategies, and measure
their effectiveness.  This paper reports on some very preliminary
results.  They mainly concern the question of when, or even whether,
to create a new thread.  This is an aspect which has so far received
little attention --- existing work has focused mainly
on load sharing rather than on thread creation.  
}


\section{Technical reports}

\reference
{Simon L Peyton Jones and Philip Wadler}
{A static semantics for \Haskell{}}
{Dept of Computing Science, University of Glasgow}
{
This paper gives a static semantics for a large subset of \Haskell{}, including
giving a translation into a language without overloading.
It is our intention to cover the complete language in due course.

One innovative aspect is the use of ideas from the second-order lambda
calculus to record type information in the program.

The paper is long (40 pages) and is more of a reference document than
a narrative one.
}

\reference
{Philip Wadler}
{A simple type inference algorithm}
{Dept of Computing Science, University of Glasgow}
{
This program is intended as a showcase for Haskell's
literate programming facility and for the monadic style
of programming.  It implements Hindley-Milner type inference.
Monads are used for parsing and to simplify ``plumbing'' in the type
checker. The monads for parsing, exceptions, and state as well
as the routines for unparsing are designed to be of general utility.
}
 
\reference{The Grasp team}
{The Glasgow Haskell I/O system}
{Dept of Computing Science, University of Glasgow, Nov 1991}
{
Most input/output systems for non-strict functional languages
feature a rather large ``operating system
The Glasgow Haskell system implements input and output
very largely within Haskell itself, without the conventional
enclosing ``printing mechanism''.  This paper explains how the 
IO system works in some detail.
}

\end{document}


