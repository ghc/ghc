\documentstyle[11pt,slpj]{article}

\newcommand{\reference}[4]{	% authors, title, details, abstract
\large
#1, {\em #2}, #3.
\normalsize
\begin{quotation}
#4
\end{quotation}
\vspace{0.2in}
}

\newcommand{\Haskell}[1]{{\sc Haskell}}

\begin{document}

\title{Abstracts of GRIP/GRASP-related design documents and manuals \\
Dept of Computing Science \\
University of Glasgow G12 8QQ}

\author{
Cordelia Hall (cvh@cs.glasgow.ac.uk) \and
Kevin Hammond (kh@cs.glasgow.ac.uk) \and
Will Partain (partain@cs.glasgow.ac.uk) \and
Simon L Peyton Jones (simonpj@cs.glasgow.ac.uk) \and
Phil Wadler (wadler@cs.glasgow.ac.uk) 
}

\maketitle

\begin{abstract}
This list covers internal design documents and manuals for the GRIP
and GRASP projects.
They are mainly intended for internal consumption, or for brave friends.

Reports and papers designed for more general consumption are given in 
a separate list.

They of them can be obtained by writing to 
Teresa Currie, Dept of Computing Science,
University of Glasgow G12 8QQ, UK.   Her electronic mail address is
teresa@uk.ac.glasgow.cs.
\end{abstract}


\section{Manuals, design documents and guides}

\reference{Kevin Hammond and Simon L Peyton Jones}
{Mail server guide}
{Nov 1990}
{
A guide to the GRIP Mail Server
}

\reference{Kevin Hammond, Simon L Peyton Jones and Jon Salkild}
{GLOS 2.0 - The GRIP Lightweight Operating System}
{University College London, January 1989}
{
GLOS is a lightweight multitasking non-preemptive operating 
for the GRIP multiprocessor.
This paper describes the operating system from the programmer's point of
view.
}

\reference{Simon L Peyton Jones and Jon Salkild}
{GRIP system user manual}
{University College London, January 1989}
{
This document describes how to configure, boot and run the GRIP system,
using the sys2 system mangement program.
}

\reference{Simon L Peyton Jones}
{The BIP front panel user manual}
{University College London, January 1989}
{
This document describes {bsim} the program which runs on the GRIP host
Unix machine, and provides a front-panel interface to the BIP.
It assumes familiarity with the BIP architecture.
}

\reference{Chris Clack}
{The GRIP Intelligent Memory Unit microprogrammer's guide}
{University College London, January 1989}
{
This paper encapsulates the spectrum of knowledge required to microprogram
the GRIP Intelligent Memory Units (IMUs). It gives a detailed 
description of the IMU hardware and its microassembler, together with 
the library of predefined microcode macros.
An overview of the the Bus Interface 
Processor (BIP) hardware and its interface protocols is also provided.
}

\reference{Chris Clack}
{Diagnostic control and simulation of GRIP Intelligent Memory Units - the
msH user guide}
{University College London, January 1989}
{
Software has been written to facilitate interaction with the diagnostic
hardware embedded in each GRIP Intelligent Memory Unit (IMU).
The msS program precisely emulates an IMU, and can be used to help
debug IMU microcode in the absence of real hardware.
The msH program interfaces directly to the actual hardware.
Both msS and msH are driven by the same interactive front panel, which 
both acts a command interpreter and manages the display screen.

The paper is mainly concerned with a description of the front-panel and
how to use it, but also gives a brief overview of the IMU architecture.
}

\end{document}
