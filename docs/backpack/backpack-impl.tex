\documentclass{article}

\usepackage{graphicx} %[pdftex] OR [dvips]
\usepackage{fullpage}
\usepackage{float}
\usepackage{titling}
\setlength{\droptitle}{-6em}

\newcommand{\ghcfile}[1]{\textsl{#1}}

\title{Implementing Backpack}

\begin{document}

\maketitle

The purpose of this document is to describe an implementation path
for Backpack~\cite{Kilpatrick:2014:BRH:2535838.2535884} in GHC\@.

We start off by outlining the current architecture of GHC, ghc-pkg and Cabal,
which constitute the existing packaging system.  We then state what our subgoals
are, since there are many similar sounding but different problems to solve.  Next,
we describe the ``probably correct'' implementation plan, and finish off with
some open design questions.  This is intended to be an evolving design document,
so please contribute!

\section{Current packaging architecture}

The overall architecture is described in Figure~\ref{fig:arch} (ignore
the red and green bits for now).

\begin{figure}[H]
    \center{\scalebox{0.8}{\includegraphics{arch.png}}}
\label{fig:arch}\caption{Architecture of GHC, ghc-pkg and Cabal. Green bits indicate additions from upcoming IHG work, red bits indicate additions from Backpack.  Orange indicates a Haskell library.}
\end{figure}

Here, arrows indicate dependencies from one component to another.
(insert architectural description here)

A particularly important component of this picture is the package
database, sketched in Figure~\ref{fig:pkgdb}.

\begin{figure}[H]
    \center{\scalebox{0.8}{\includegraphics{pkgdb.png}}}
\label{fig:pkgdb}\caption{Anatomy of a package database and a package identifier.}
\end{figure}

An installed package is calculated from a Cabal file through the process
of dependency resolution and compilation.  We can think of it a
database, whose primary key is the InstalledPackageId, which presently
is the package name, the package version, and the ABI hash (nota bene,
the diagram disagrees with this text: it shows the version of installed
package IDs which we'd like to move towards.)  These IDs uniquely
identify an instance of an installed package.  A mere PackageId omits
the ABI hash, and is used to qualify linker exported symbols: this is
communicated to GHC using the \verb|-package-id| flag.

The database entry itself contains the information from the installed package ID,
as well as information such as what dependencies it was linked against, where
its compiled code and interface files live, its compilation flags, what modules
it exposes, etc.  Much of this information is only relevant to Cabal; GHC
uses a subset of the information in the package database.

\section{Goals}

There are actually a number of different goals we have for modifying the
packaging system.

\begin{itemize}
    \item Support multiple instances of containers-2.9 \emph{in the
        package database}.  These instances may be compiled against
        different dependencies, have the same dependencies but different
        source files (as when a package is being developed), or be
        compiled with different options.  It is less important to allow
        these instances to be linkable together.

    \item Some easy-to-implement subset of the functionality provided by
        packages with holes (Backpack proper).
\end{itemize}

A lower priority goal is to actually allow multiple instances of
containers-2.9 to be linked together in the same executable
program.\footnote{In particular, this requires changes to how linker symbols
are assigned. However, this feature is important to implement a number
of Backpack features.}

A \emph{non-goal} is to allow users to upgrade upstream libraries
without recompiling downstream. This is an ABI concern and we're not
going to worry about it.

\section{Aside: Recent IHG work}\label{sec:ihg}

The IHG project has allocated some funds to relax the package instance
constraint in the package database, so that multiple instances can be
stored, but now the user of GHC must explicitly list package-IDs to be
linked against.  In the far future, it would be expected that tools like
Cabal automatically handle instance selection among a large number of
instances, but this is subtle and so this work is only to do some
foundational work, allowing a package database to optionally relax the
unique package-version requirement, and utilize environment files to
select which packages should be used.  See Duncan's email for more
details on the proposal.

For the purpose of Backpack, the only relevant part of this proposal
is the relaxation of package databases so that there is no uniqueness
constraint on PackageIds; only InstalledPackageIds are unique.

To implement this:

\begin{enumerate}

    \item Remove the ``removal step'' when registering a package (with a flag)

    \item Check \ghcfile{compiler/main/Packages.lhs}:mkPackagesState to look out for shadowing
      within a database. We believe it already does the right thing, since
      we already need to handle shadowing between the local and global database.

\end{enumerate}

Once these changes are implemented, we can program multiple instances by
using \verb|-hide-all-packages -package-id ...|, even if there is no
high-level tool support.

\section{Adding Backpack to GHC}

Backpack additions are described in red in the architectural diagrams.
The current structure of this section is to describe the additions bottom up.

\subsection{Physical identity = InstalledPackageId + Module name}\label{sec:ipi}

In Backpack, there needs to be some mechanism for assigning
\emph{physical module identities} to modules, which are essential for
typechecking Backpack packages, since they let us tell if two types are
equal or not. In the paper, the physical identity was represented as the
package that constructed it as well as some representation of the module
source.  We can simplify this slightly: in current Cabal packages, we
require that modules always be given a package-unique logical name;
thus, physical identities can be simply represented as a PackageId plus
module name. (See \ghcfile{compiler/basicTypes/Module.lhs:Module})

However, this is not enough if we allow multiple instances of a package
in the package database: now we may incorrectly conclude that types
defined by different instances of the same version of a package are
equal: this would be especially fatal if the two packages were linked
against different underlying libraries.  Thus, a physical module name
should be represented as an InstalledPackageId (which uniquely
identifies an installed package) as well as the original logical name
(bottom of Figure~\ref{fig:pkgdb}).

To implement Backpack, we need to change the way GHC internally represents
module to qualify these using InstalledPackageId, not PackageId.  There
is also some user-visible changes: when GHC compiles code, it does so
under a \emph{current PackageId} specified by \verb|-package-name|.  A
new flag must be added to specify what the current InstalledPackageId
is.  But see also the caveats below.

\paragraph{Note about linker symbols} Currently the \verb|-package-name|
option is used both for typechecking, and then in CLabels which are used
to assign exported linker symbols (e.g.
\verb|base_TextziReadziLex_zdwvalDig_info|).  However, we don't really
want to use InstalledPkgId to generate linker names, because whenever
the extra unique signature changes, all of the exported linker names
would also change, ensuring that nothing is ever ABI compatible, ever.
One approach is to only use the InstalledPackageId for type-checking,
and then use only PackageId for linker name generation.  So, it probably
makes sense to use the old linker behavior in the short term.

\paragraph{Note about opaqueness of InstalledPackageId}  Currently,
InstalledPackageId is an opaque string which is allocated by Cabal; GHC
never parses these identifiers to determine metadata about the package
in question.  So, if we want to preserve old exported symbols behavior,
we still need to provide a PackageId via \verb|package-name|, so that an
appropriate name can be output.

\paragraph{Note about using the ABI hash} Currently, InstalledPackageId
is constructed of a package, version and ABI hash
(generateRegistrationInfo in
\ghcfile{libraries/Cabal/Cabal/Distribution/Simple/Register.hs}).  The
use of an ABI hash is a bit of GHC-specific hack introduced in 2009,
intended to make sure these installed package IDs are unique.  While
this is quite clever, using the ABI is actually a bit inflexible, as one
might reasonably want to have multiple copies of a package with the same
ABI but different source code changes.\footnote{In practice, our ABIs
are so unstable that it doesn't really matter.}

In Figure~\ref{fig:pkgdb}, there is an alternate logical representation
of InstalledPackageId which attempts to extricate the notion of ABI
compatibility from what actually might uniquely identify a package.
We imagine these components to be:

\begin{itemize}
    \item The package and version, as before;
    \item A hash of the source code (so one can register different
        in-development versions without having to bump the version
        number);
    \item Compilation flags (such as compilation way, optimization,
        profiling settings)\footnote{This is a little undefined on a package bases, because in principle the flags could be varied on a per-file basis. More likely this will be approximated against the relevant fields in the Cabal file as well as arguments passed to Cabal.};
    \item InstalledPackageIds of dependencies that were linked against.
\end{itemize}

It's also important to not use ABI, because we don't know what the ABI
is until after we compile, but when I'm using it for typechecking, I'm
obligated to provide \emph{some} InstalledPackageId from the get-go.

A historical note: in the 2012 GSoC project to allow multiple instances
of a package to be installed at the same time, use of \emph{random
numbers} was used to workaround the inability to get an ABI early
enough.  This seemed a bit dodgy, so we're not going to do that here.

\paragraph{Wired-in names} One annoying thing to remember is that GHC
has wired-in names, which refer to packages without any version.  A
suggested approach is to have a fixed table from these wired names to
package IDs.

\subsection{Exposed modules should allow external modules}\label{sec:reexport}

In Backpack, the definition of a package consists of a logical context,
which maps logical module names to physical module names.  These do not
necessarily coincide, since some physical modules may have been defined
in other packages and mixed into this package.  This mapping specifies
what modules other packages including this package can access.
However, in the current installed package database, we have exposed-modules which
specify what modules are accessible, but we assume that the current
package is responsible for providing these modules.

To implement Backpack, we have to extend this exposed-modules (``Export declarations''
on Figure~\ref{fig:pkgdb}).  Rather
than a list of logical module names, we provide a new list of tuples:
the exported logical module name and original physical module name (this
is in two parts: the InstalledPackageId and the original module name).
For example, an traditional module export is simply (Name, my-pkg-id, Name);
a renamed module is (NewName, my-pkg-id, OldName), and an external module
is (Name, external-pkg-id, Name).

\subsection{Indefinite packages}

In Backpack, some packages still have holes in them, to be linked in later.
GHC cannot compile these packages, but we still need to install them because
other packages may still type-check against them, and eventually we will
need to compile them (once some downstream package links it against its
dependencies.)

It seems clear that we need to install packages which do not contain
compiled code, but have all of the ingredients necessary to compile them.
We imagine that instead of providing path to object files, an \emph{indefinite
package} which contains just interface files as well as source. (Figure~\ref{fig:pkgdb})

Creating and typechecking single instances of indefinite packages seems to
be unproblematic: GHC can already just type-check code (without compiling it),
and we can also type-check against an interface file, which is currently used for
the recursive module, hs-boot mechanism. (Figure~\ref{fig:arch})

When we need to compile an indefinite package (since all of its
dependencies have been found), things get a bit knotty.  In particular,
there seem to be two implementation paths for this compilation: one path
closer to how GHC compilation currently works, and another which is
conceptually closer to the Backpack formalism.  Here is a very simple
example to consider for both cases:

\begin{verbatim}
package pkg-a where
    A = ...
package pgk-b where -- indefinite package
    A :: ...
    B = [ b = ... ]
package pkg-c where
    include pkg-a
    include pkg-b
\end{verbatim}

\paragraph{The ``downstream'' proposal}  At some point, a package which
relies on an indefinite package fills in all of its dependencies, so
that it can be compiled.  Compilation proceeds by treating all of the
uncompiled indefinite packages as part of a single package: the current
package.  We maintain the invariant that any code generated will export
symbols under the current package's namespace.  So the identifier
\verb|b| in the example becomes a symbol \verb|pkg-c_pkg-b_B_b| rather
than \verb|pkg-b_B_b| (package subqualification is necessary because
package C may define its own B module after thinning out the import.)

One big problem with this proposal is that it doesn't implement applicative
semantics.  If there is another package:

\begin{verbatim}
package pkg-d where
    include pkg-a
    include pkg-b
\end{verbatim}

this will generate its own instance of B, even though it should be the same
as C.  Simon was willing to entertain the idea that, well, as long as the
type-checker is able to figure out they are the same, then it might be OK
if we accidentally generate two copies of the code (provided they actually
are the same).

\paragraph{The ``upstream'' proposal}  Instead of treating all
uncompiled indefinite packages as a single package, each fully linked
package is now considered an instance of the original indefinite
package, except its dependencies are filled in further.

One big change that is necessary is that we must augment exported
linker symbols to include a hash, or some serial number into a registry,
of the true physical module identity of linked modules, which will
generally be some recursive tree.  Then identifier \verb|b| becomes
\verb|pkg-b-HASH-b_B|, where HASH represents the physical module
identity.  These instantiations of packages are hash-consed, so if
someone else constructs the exact same dependency change, the instance
will be reused.

\paragraph{Aliases} There are some problems with respect to what occurs when two
distinct signatures are linked together (aliasing), we talk these problems in
Section~\ref{sec:open-questions}.

\paragraph{Aside: Original names} Original names are an important design pattern
in GHC\@.
Sometimes, a name can be exposed in an hi file even if its module
wasn't exposed. Here is an example (compiled in package R):

\begin{verbatim}
module X where
    import Internal (f)
    g = f

module Internal where
    import Internal.Total (f)
\end{verbatim}

Then in X.hi:

\begin{verbatim}
g = <R.id, Internal.Total, f> (this is the original name)
\end{verbatim}

(The reason we refer to the package as R.id is because it's the
full package ID, and not just R).

How might internal names work with Backpack?

\begin{verbatim}
package P where
    M = ...
    N = ...
package Q (M, R, T)
    include P (N -> R)
    T = ...
\end{verbatim}

And now if we look at Q\@:

\begin{verbatim}
exposed-modules:
        M -> <P.id, M>
        R -> <P.id, N>
        T -> <Q.id, T>
\end{verbatim}

When we compile Q, and the interface file gets generated, we have
to generate identifiers for each of the exposed modules.  These should
be calculated to directly refer to the ``original name'' of each them;
so for example M and R point directly to package P, but they also
include the original name they had in the original definition.


\section{Open questions}\label{sec:open-questions}

Here are open problems about the implementation which still require
hashing out.

\begin{itemize}
    \item Aliasing of signatures means that it is no longer the case that
      original name means type equality.  We were not able to convince
      Simon of any satisfactory resolution.  Strawman proposal is to
      extending original names to also be variables probably won't work
      because it is so deeply wired, but it's difficult to construct hi
      files so that everything works out (quadratic).

  \item Relationship between linker names and InstalledPackageId? The reason
      the obvious thing to do is use all of InstalledPackageId for linker
      name, but this breaks recompilation.  So some of these things
      should go in the linker name, and not others (yes package, yes
      version, yes some compile flags (definitely ways), eventually
      dependency package IDs, what about cabal build flags).  This is
      approximately an ABI hash, but it's computable before compilation.
      This worries Simon because now there are two names, but maybe
      the DB can solve that problem---unfortunately, GHC doesn't ever
      register during compilation; only later.

        Simon also thought we should use shorter names for linker
        names and InstallPkgIds.  This appears to be orthogonal.

    \item In this example:

\begin{verbatim}
    package A where
        A = ...
    package A2 where
        A2 = ...
    package B (B)
        A :: ...
        B = ...
    package C where
        include A
        include B
    package D where
        include A
        include B
    package E where
        include C (B as CB)
        include D (B as DB)
\end{verbatim}

      Do the seperate instantiations of B exist as seperate artifacts
      in the database, or do they get constructed on the fly by
      definite packages which include them?  The former is conceptually
      nicer (identity of types work, closer to Backpack semantics), but
      the latter is easier for linker names. (Simon's first inclination
      is to construct it on the fly.)

      There was another example, showing that we need to solve this
      problem even for indefinite combinations of indefinite modules.
      You can get to it by modifying the earlier example so that C and
      D still have holes, which E does not fill.

  \item We have to store the preprocessed sources for indefinite packages.
      This is hard when we're constructing packages on the fly.

  \item What is the impact on dependency solving in Cabal?  Old questions
      of what to prefer when multiple package-versions are available
      (Cabal originally only needed to solve this between different
      versions of the same package, preferring the oldest version), but
      with signatures, there are more choices.  Should there be a
      complex solver that does all signature solving, or a preprocessing
      step that puts things back into the original Cabal version.
      Authors may want to suggest policy for what packages should actually
      link against signatures (so a crypto library doesn't accidentally
      link against a null cipher package).
      \end{itemize}

\section{Immediate tasks}

At this point in time, it seems we can identify the following set
of non-controversial tasks which can be started immediately.

\begin{itemize}
    \item Relax the package database constraint to allow multiple
        instances of package-version. (Section~\ref{sec:ihg})
    \item Propagate the use of \verb|InstalledPackageId| instead of
        package IDs for typechecking. (Section~\ref{sec:ipi})
    \item Implement export declarations in package format, so
        packages can reexport modules from other packages. (Section~\ref{sec:reexport})
\end{itemize}

The aliasing problem is probably the most important open problem
blocking the rest of the system.

\bibliographystyle{plain}
\bibliography{backpack-impl}

\end{document}
